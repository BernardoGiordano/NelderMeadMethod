\documentclass[a4paper, 11pt]{article}
\usepackage{fullpage}
\usepackage{mathtools, nccmath}
\usepackage{graphicx}
\usepackage{float}
\usepackage{amsmath}
\usepackage{caption}
\renewcommand{\figurename}{Fig.}
\renewcommand{\refname}{Bibliografia}

\begin{document}
% Header
\noindent
\large\textbf{Corso di Metodi di Ottimizzazione} \hfill \textbf{Gruppo E} \\
\normalsize A.A. 2018/2019 \hfill Ing. Saverio Del Prete \\
Prof. Raffaele Martone \hfill Ing. Bernardo Giordano \\
\hphantom{}\hfill Ing. Lucia Migliaccio

\section*{Traccia del problema}

Progetto ottimo di un campo magnetico con incognite geometriche e di corrente di
una spira.

\section*{Descrizione del sistema fisico}

Il sistema è composto da un certo numero di spire simmetriche (supporremo $n=6$)
concentriche rispetto allo stesso asse $z$. I parametri di progetto, ovvero
posizione, raggio e intensità di corrente, sono noti per tutte le spire tranne
che per una coppia. Le correnti sono concordi per tutte le spire. 

\begin{figure}[H]
    \centering
    \includegraphics[width=12cm]{assets/figure1}
    \caption{Schematizzazione del sistema fisico.}
\end{figure}

\noindent
Il campo magnetico generato dalla corrente circolante nelle spire può essere
valutato sull’asse utilizzando la legge di Biot-Savart. La legge di Biot-Savart
permette di valutare il campo magnetico B prodotto in un punto dello spazio da
una spira percorsa da corrente elettrica. \\ \\
\noindent
Il campo magnetico sull’asse $z$  di una spira caratterizzata da una corrente
$I$, lunghezza $L$, raggio $R$ e posizione $Z$ si valuta come
\[dB_{z}=\frac{\mu_{0}IdL}{4\pi}\frac{R}{\left((Z{\pm}z)^{2}+R^{2}\right)^{3/2}}\]
dove $\mu_{0}$ è la permeabilità magnetica nel vuoto, $\mu_{0}=4{\pi}10^{-7}
H/m$. \\
L’integrale di $dL$ risulta essere proprio la circonferenza della spira, ovvero
$2{\pi}R$. Integrando quindi, si ricava la funzione del campo magnetico
sull'asse
\[B_{z}=\frac{\mu_{0}}{4\pi}\frac{2{\pi}R^{2}I}{\left((Z{\pm}z)^{2}+R^{2}\right)^{3/2}}=\frac{\mu_{0}}{2}\frac{R^{2}I}{\left((Z{\pm}z)^{2}+R^{2}\right)^{3/2}}\]
Siano $Z_{i}$, $R_{i}$ e $I_{i}$ rispettivamente la posizione, il raggio e la
corrente relative alla spira $i$-esima. \\
Sia inoltre $n=6$ il numero complessivo delle spire facenti parte del sistema.
Considerando adesso la sovrapposizione degli effetti di tutte le spire del
sistema e tenendo presente che le spire sono simmetriche rispetto al piano
$r{\theta}$, il campo magnetico complessivo sull’asse $z$ sarà:

\begin{align*}
    B_{z}
    &=\frac{\mu_{0}}{4{\pi}}\left(\sum_{i=1}^{n}\frac{2{\pi}I_{i}R_{i}^{2}}{\left(\left(Z_{i}-z\right)^{2}+R_{i}^{2}\right)^{3/2}}+\frac{2{\pi}I_{i}R_{i}^{2}}{\left(\left(Z_{i}+z\right)^{2}+R_{i}^{2}\right)^{3/2}}\right)\\
    &=\frac{\mu_{0}}{2}\left(\sum_{i=1}^{n}\frac{I_{i}R_{i}^{2}}{\left(\left(Z_{i}-z\right)^{2}+R_{i}^{2}\right)^{3/2}}+\frac{I_{i}R_{i}^{2}}{\left(\left(Z_{i}+z\right)^{2}+R_{i}^{2}\right)^{3/2}}\right)
\end{align*}

\noindent
L'obiettivo del progetto della spira mancante risulta essere la migliore
approssimazione di un campo magnetico avente la seguente caratteristica:

\begin{figure}[H]
    \centering
    \includegraphics[width=16cm]{assets/figure2}
    \caption{Campo magnetico sull'asse $z$ desiderato.}
\end{figure}
\noindent
Sia adesso $B_{z}$ il campo magnetico generato dalla sovrapposizione delle spire
del sistema, e $\tilde{B_{z}}$ il campo magnetico desiderato. Al fine di
ottenere una discrepanza più piccola possibile, è di particolare interesse lo
studio del minimo della funzione
\[||B_{z}-\tilde{B_{z}}||\] Siccome i parametri di progetto delle spire 1 e 3
del sistema (e, quindi, delle rispettive spire simmetriche) sono noti ed
esattamente uguali a quelli del sistema desiderato, la suddetta differenza in
norma si scriverà come:

\begin{equation}
    \begin{split}
        ||B_{z_{2}}-\tilde{B_{z_{2}}}||
        =& \Bigl|\Bigl|\frac{\mu_{0}}{2}(\frac{I_{2}R_{2}^{2}}{((Z_{2}-z)^{2}+R_{2}^{2})^{3/2}}+\frac{I_{2}R_{2}^{2}}{((Z_{2}+z)^{2}+R_{2}^{2})^{3/2}})-\tilde{B_{z_{2}}}\Bigr|\Bigr| \\
        =& \frac{\mu_{0}}{2}\Bigl|\Bigl|\frac{I_{2}R_{2}^{2}}{((Z_{2}-z)^{2}+R_{2}^{2})^{3/2}}+\frac{I_{2}R_{2}^{2}}{((Z_{2}+z)^{2}+R_{2}^{2})^{3/2}}-\frac{2}{\mu_{0}}\tilde{B_{z_{2}}}\Bigr|\Bigr|
    \end{split} 
\end{equation}

\noindent
La funzione che quindi descrive la discrepanza del campo magnetico da
progettaree quello desiderato su tutto l'asse $z$ è 

\[F_{obj}=2\int_{0}^{+\infty}|B_{z_{2}}-\tilde{B_{z_{2}}}|dz\]

\[F_{obj}=\frac{1}{N_{p}}\sum\limits_{k=1}^{Np}
||B_{z_{2k}}-\tilde{B_{z_{2k}}}||\]

\section*{Campionamento e normalizzazione}

Prima di poter definire la funzione obiettivo ed effettuarne un'analisi
numerica, è necessario applicare due tecniche: campionamento e normalizzazione.
\\
Il campionamento è una tecnica che consiste nel discretizzare una funzione
continua nel tempo. L'ampiezza del passo di campionamento può essere valutato a
intervalli temporali regolari e non. \\
In questo caso specifico, si è deciso di utilizzare un campionamento a passo
costante. \\
La normalizzazione è l'operazione la quale, dato un vettore, lo si porta ad
avere una norma unitaria. \\
% aggiungere altre cose sulla normalizzazione
La funzione verrà campionata su $N_{p}$ valori dell'asse $z$. La funzione
campionata, inoltre, verrà normalizzata e mediata sul valor medio di
$\tilde{B_{z}}$, in modo tale da esprimere la funzione obiettivo come
percentuale di scostamento tra il campo magnetico progettato e quello ideale. \\
Campionando e normalizzando la (1), la \emph{funzione obiettivo} si scriverà
come:
 
\begin{equation}
    \begin{split}
        F_{obj}(I_{2},R_{2},Z_{2}) =
        & \frac{1}{{\tiny <} \tilde{B_{z}}{\tiny >}} \Biggl[ \frac{1}{Np} \sum\limits_{k=1}^{Np} \Bigl|\Bigl|\frac{I_{2}R_{2}^{2}}{((Z_{2}-z_{k})^{2}+R_{2}^{2})^{3/2}}+\frac{I_{2}R_{2}^{2}}{((Z_{2}+z_{k})^{2}+R_{2}^{2})^{3/2}}-\tilde{B^*_{z_{2k}}}\Bigr|\Bigr|^{2}\biggr]^{\!1/2}
    \end{split} 
\end{equation}
\noindent

\section* {Ricerca del minimo}

L'algoritmo di ricerca del minimo fa uso del concetto di simplesso, cioè un
politopo di $N+1$ vertici in $N$ dimensioni, vale a dire: un segmento in una
dimensione, un triangolo in due dimensioni, un tetraedro in tre dimensioni, e
così via. Il metodo permette di limitare la ricerca della soluzione ottima ai
vertici del politopo. \\
La ricerca avviene attraverso il movimento del politopo, il quale può:

\begin{itemize}
\item \textbf{Ribaltarsi}: si valuta la funzione negli $N$ vertici del
simplesso, individuando qual è il vertice nel quale la funzione assume il valore
\emph{peggiore} (nel caso di un problema di ricerca del minimo, il caso peggiore
è quello in cui la funzione assume valore più grande tra tutti i vertici del
simplesso considerato). Una volta identificato il vertice peggiore, il simplesso
si ribalta rispetto a quest'ultimo, tenendo fermi gli altri $N-1$ vertici. \\
L'algoritmo può però ritrovarsi in una situazione di loop, dove il vertice
peggiore risulta essere proprio l'ultimo rispetto al quale è stato effettuato il
ribaltamento. In questo caso, si ribalta rispetto al \emph{secondo peggior
vertice}. 
\item \textbf{Contrarsi}: se un vertice del politopo \emph{vive} più a lungo di
un numero arbitrario di iterazioni $M$, il politopo viene contratto, dimezzando
i lati dell'ultimo simplesso tenendo fermo il vertice \emph{migliore}.
\end{itemize}

\noindent
L'algoritmo di ricerca del minimo mediante l'uso del politopo è un metodo di
\emph{ordine 0}, ovvero non richiede l'uso delle derivate, nonostante riesca a
stimare una direzione di discesa della funzione obiettivo.

\section*{Implementazione dell'algoritmo}

L'algoritmo implementato per la soluzione del problema della ricerca del minimo
consiste in una fase di \emph{inizializzazione}, una di \emph{loop} (nella quale
vengono effettuate tutte le operazioni di ribaltamento e contrazione del
simplesso) e una di \emph{controllo delle condizioni di arresto} del loop. \\ \\
La fase di inizializzazione consiste nell'impostazione di tutte le variabili
necessarie al funzionamento dell'algoritmo, che per design vengono
parametrizzate, in modo tale da rendere comodo l'utilizzo dell'algoritmo in
differenti condizioni iniziali, tra cui:
\begin{itemize}
\item \textbf{Passo di campionamento}, necessario all'approssimazione più o meno
grossolana della funzione obiettivo da studiare.
\item \textbf{Range dei parametri liberi}, con i quali si stabiliscono due
valori limite che dovranno essere rispettati da tutte le variabili da
ottimizzare.
\item \textbf{Condizioni di arresto}, necessarie per stabilire fino a quanto
l'algoritmo dovrà iterare per garantire dei risultati soddisfacenti. Tra le
innumerevoli condizioni possibili, verranno considerate il \emph{massimo numero
di ribaltamenti}, un valore di \emph{massimo errore percentuale} rispetto alla
funzione desiderata e un vincolo sulla \emph{minima lunghezza del lato} del
simplesso generato ad ogni passi.
\item \textbf{Punto iniziale}, che condizionerà, insieme all'andamento della
funzione, la generazione di determinati simplessi piuttosto che altri.
\end{itemize}

\noindent
Una volta parametrizzate le condizioni iniziali, l'algoritmo entrerà in fase di
loop, dove eseguirà ripetutamente le seguenti operazioni:

\begin{enumerate}
\item Un simplesso iniziale $s$ con centroide in $s_{0}$ viene aggiunto ad un
array di simplessi $polytope$
\item Controllo se un vertice di $s = polytope(end)$ ha vissuto per $k$
iterazioni
\item Se si:
\begin{enumerate}
\item Trovo il vertice $i$ di $s$ dove $F_{obj}$ assume valore minore
\item Dimezzo la lunghezza dei lati di $s$ lasciando fermo il vertice $i$,
ottenendo $s_{new}$
\end{enumerate}
\item Se no:
\begin{enumerate}
\item Trovo il vertice $i$ di $s$ dove $F_{obj}$ assume valore massimo
\item Ribalto $s$ tenendo fermi i vertici $k \ne i$, ottenendo $s_{new}$
\item Se $F_{obj}(s(i)) \leq F_{obj}(s_{new}(i))$:
\begin{enumerate}
\item Trovo il secondo vertice $j$ di $s$ dove $F_{obj}$ assume valore massimo
\item Ribalto $s$ tenendo fermi i vertici $k \ne j$, ottenendo $s_{new}$
\end{enumerate}
\end{enumerate}
\item Aggiungo $s_{new}$ a $polytope$, $polytope(end+1) = s_{new}$
\item Interrompi se almeno una condizione di arresto è verificata, altrimenti
ripeti dal passo 2.
\end{enumerate}

\noindent
L'algoritmo può essere schematizzato in maniera compatta secondo il seguente
diagramma di flusso:

\begin{figure}[H]
    \centering
     \includegraphics[width=14cm]{assets/figure3}
     \caption{Diagramma di flusso dell'algoritmo del simplesso.}
\end{figure}
\noindent

\section*{Scelta dei parametri}

Al fine di avvicinarci alla caratteristica descritta in $Fig. 2$, vanno scelti
opportunamente il valore dei raggi, delle correnti e delle posizioni relative a
tutte le spire, comprese quelle da progettare (spire 2 e 4). \\ \\
\centerline{ \textbf{Spira 1}: $R_{1}$ = 0.7m \\ $I_{1}$ = 3 A \\ $Z_{1}$ = -0.4m} 
\centerline{ \textbf{Spira 2}: $R_{2}$ = 0.8m \\ $I_{2}$ = 5 A \\ $Z_{2}$ = -0.7m}
\centerline{ \textbf{Spira 3}: $R_{3}$ = 0.6m \\ $I_{3}$ = 2 A \\ $Z_{3}$ = -0.9m}
\centerline{ \textbf{Spira 4}: $R_{4}$ = 0.7m \\ $I_{4}$ = 3 A \\ $Z_{4}$ = 0.4m}
\centerline{ \textbf{Spira 5}: $R_{5}$ = 0.8m \\ $I_{5}$ = 5 A \\ $Z_{5}$ = 0.7m}
\centerline{ \textbf{Spira 6}: $R_{6}$ = 0.6m \\ $I_{6}$ = 2 A \\ $Z_{6}$ = 0.9m}
\\ \\
\noindent
Si terrà inoltre considerazione di un campionamento a passo costante e di un
range di campionamento dell'asse $z$ che si estenderà fino a circa 3 volte oltre
la posizione dell'ultima spira. Siccome la spira più lontana dal centro di
simmetria è posta in $z = 0.9 m$, l'asse $z$ verrà campionato per $-3 m \le z
\le 3 m$. \\
Limiteremo a priori, inoltre, il massimo raggio e la massima posizione della
spira da progettare, in modo tale che entrambe non superino $1 m$.

\section*{Esperimenti}
\section{Minimo in 2 dimensioni senza vincoli}

Il primo test effettuato è stata la ricerca del minimo con due gradi di libertà,
supponendo che la variabile $Z = 0.7m$ sia fissata. Otteniamo i seguenti
risultati:

\begin{table}[h]
\caption{Test 1}
\begin{center}
\begin{tabular}{|l|l|l|l|} 
\hline
\multicolumn{2}{|c|}{\textbf{Parametri in ingresso}} &
\multicolumn{2}{c|}{\textbf {Valori in uscita}} \\ \hline
Gradi di libertà  & \textbf{2} &  &  \\ \hline 
Passo & \textbf{0.100} & Campioni & \textbf{61} \\ \hline 
Punto di minimo & \textbf{{[}5.000 0.800{]}} & Punto di minimo &
\textbf{{[}5.000 0.800{]}} \\ \hline 
Punto iniziale & \textbf{{[}3.000 0.300{]}} & Fobj nel punto & \textbf{0.00001}
\\ \hline 
Lunghezza iniziale & \textbf{0.300} & Dimezzamenti & \textbf{14} \\ \hline 
Lunghezza minima & \textbf{0.00001} & Lunghezza finale & \textbf{0.00003} \\
\hline
Flip massimi & \textbf{1000} & Flips & \textbf{177} \\ \hline 
Tolleranza & \textbf{0.00100\%} & \textbf{Perc. di errore} & \textbf{0.00077\%}
\\ \hline 
\end{tabular} 
\end{center}
\end{table}

\begin{figure}[H]
    \centering
        \includegraphics[width=15cm]{assets/figure10}
        \caption{Test con due gradi di libertà senza vincoli.}
\end{figure}

\section{Minimo in 3 dimensioni senza vincoli}

Proponiamo adesso dei test di ricerca del minimo non vincolato con 3 gradi di
libertà. La percentuale di errore diminuisce all'aumentare del numero di
campioni, a parità di percentuale di errore massima.

\noindent
Come si evince dai risultati, l'algoritmo si arresta con una percentuale di
errore in uscita minore rispetto al caso con passo di campionamento meno fitto.
Il punto di minimo trovato dall'algoritmo è lo stesso in quanto, a causa degli
arrotondamenti dovuti ad un limitato numero di cifre significative apprezzabili,
non possiamo andare oltre la terza.

\begin{table}[h]
    \caption{Test 1}
    \begin{center}
    \begin{tabular}{|l|l|l|l|} 
    \hline 
\multicolumn{2}{|c|}{\textbf{Parametri in ingresso}} &
\multicolumn{2}{c|}{\textbf{Valori in uscita}} \\ \hline
Gradi di libertà  & \textbf{3} &  &  \\ \hline 
Passo & \textbf{0.100} & Campioni & \textbf{61} \\ \hline 
Punto di minimo & \textbf{{[}5.000 0.800 0.700{]}} & Punto di minimo &
\textbf{{[}4.999 0.799 0.700{]}} \\ \hline 
Punto iniziale & \textbf{{[}3.000 0.300 0.500{]}} & Fobj nel punto &
\textbf{0.00004} \\ \hline 
Lunghezza iniziale & \textbf{0.500} & Dimezzamenti & \textbf{16} \\ \hline 
Lunghezza minima & \textbf{0.00001} & Lunghezza finale & \textbf{0.00001} \\
\hline
Flip massimi & \textbf{1000} & Flips & \textbf{654} \\ \hline 
Tolleranza & \textbf{0.00100\%} & \textbf{Perc. di errore} & \textbf{0.00417\%}
\\ \hline 
    \end{tabular}
    \end{center}
    \end{table}

\begin{table}[h]
    \caption{Test 2}
    \begin{center}
    \begin{tabular}{|l|l|l|l|} 
    \hline 
\multicolumn{2}{|c|}{\textbf{Parametri in ingresso}} &
\multicolumn{2}{c|}{\textbf{Valori in uscita}} \\ \hline
Gradi di libertà  & \textbf{3} &  &  \\ \hline 
Passo & \textbf{0.0100} & Campioni & \textbf{601} \\ \hline 
Punto di minimo & \textbf{{[}5.000 0.800 0.700{]}} & Punto di minimo &
\textbf{{[}4.999 0.799 0.700{]}} \\ \hline 
Punto iniziale & \textbf{{[}3.000 0.300 0.500{]}} & Fobj nel punto &
\textbf{0.00002} \\ \hline 
Lunghezza iniziale & \textbf{0.500} & Dimezzamenti & \textbf{16} \\ \hline 
Lunghezza minima & \textbf{0.00001} & Lunghezza finale & \textbf{0.00001} \\
\hline
Flip massimi & \textbf{1000} & Flips & \textbf{709} \\ \hline 
Tolleranza & \textbf{0.00100\%} & \textbf{Perc. di errore} & \textbf{0.00205\%}
\\ \hline 
    \end{tabular}
    \end{center}
    \end{table}

\begin{figure}[H]
    \centering
        \includegraphics[width=16cm]{assets/figure4}
        \caption{Ricerca del minimo non vincolato con 3 parametri liberi, test 5.}
\end{figure}
\noindent 

\begin{figure}[H]
    \centering
        \includegraphics[width=14cm]{assets/figure5}
        \caption{Dettaglio del politopo generato al test 5.}
\end{figure}
\noindent 

\begin{figure}[H]
    \centering
        \includegraphics[width=14cm]{assets/figure6}
        \caption{Dettaglio del politopo generato al test 5.}
\end{figure}
\noindent

\newpage
\section{Minimo in 3 dimensioni con vincolo di disuguaglianza}

Effettuiamo un test per verificare il corretto funzionamento dell'algoritmo e
appurarne le prestazioni in presenza di un vincolo di disuguaglianza $R \le 2Z$.
\\
Il vincolo di disuguaglianza è stato trattato algoritmicamente con la tecnica
delle \emph{penalità}, in modo tale da rendere molto alto il valore della
funzione obiettivo valutato nei vertici del simplesso oltrepassanti il vincolo
(mediante un fattore moltiplicativo $p = 100$). \\
Come si evince dai due test effettuati con gli stessi parametri ma con un punto
iniziale diverso, è evidente che i risultati sono anche molto diversi a seconda
del punto.

\begin{table}[h]
    \caption{Test 1}
    \begin{center}
    \begin{tabular}{|l|l|l|l|} 
    \hline 
\multicolumn{2}{|c|}{\textbf{Parametri in ingresso}} &
\multicolumn{2}{c|}{\textbf {Valori in uscita}} \\ \hline
Gradi di libertà  & \textbf{3} &  &  \\ \hline 
Passo & \textbf{0.100} & Campioni & \textbf{61} \\ \hline 
Punto di minimo & \textbf{{[}5.000 0.800 0.700{]}} & Punto di minimo &
\textbf{{[}4.0926 0.999 0.495{]}} \\ \hline 
Punto iniziale & \textbf{{[}4.000 0.400 0.100{]}} & Fobj nel punto &
\textbf{0.320} \\ \hline 
Lunghezza iniziale & \textbf{0.500} & Dimezzamenti & \textbf{16} \\ \hline 
Lunghezza minima & \textbf{0.00001} & Lunghezza finale & \textbf{0.00001} \\
\hline
Flip massimi & \textbf{10000} & Flips & \textbf{214} \\ \hline 
Tolleranza & \textbf{0.00100\%} & \textbf{Perc. di errore} & \textbf{32.0113\%}
\\ \hline 
    \end{tabular}
    \end{center}
    \end{table}

\begin{figure}[H]
    \centering
        \includegraphics[width=14cm]{assets/figure7}
        \caption{Test con vincolo attivo di disuguaglianza.}
\end{figure}
    
\begin{table}[h]
    \caption{Test 2}
    \begin{center}
    \begin{tabular}{|l|l|l|l|} 
    \hline 
\multicolumn{2}{|c|}{\textbf{Parametri in ingresso}} &
\multicolumn{2}{c|}{\textbf {Valori in uscita}} \\ \hline
Gradi di libertà  & \textbf{3} &  &  \\ \hline 
Passo & \textbf{0.100} & Campioni & \textbf{61} \\ \hline 
Punto di minimo & \textbf{{[}5.000 0.800 0.700{]}} & Punto di minimo &
\textbf{{[}3.407 0.885 0.443{]}} \\ \hline 
Punto iniziale & \textbf{{[}3.000 0.400 0.100{]}} & Fobj nel punto &
\textbf{0.449} \\ \hline 
Lunghezza iniziale & \textbf{0.500} & Dimezzamenti & \textbf{16} \\ \hline 
Lunghezza minima & \textbf{0.00001} & Lunghezza finale & \textbf{0.00001} \\
\hline
Flip massimi & \textbf{10000} & Flips & \textbf{233} \\ \hline 
Tolleranza & \textbf{0.00100\%} & \textbf{Perc. di errore} & \textbf{44.891\%}
\\ \hline 
    \end{tabular}
    \end{center}
    \end{table}

\begin{figure}[H]
    \centering
        \includegraphics[width=16cm]{assets/figure8}
        \caption{Test con vincolo attivo di disuguaglianza.}
\end{figure}
\noindent

\begin{figure}[H]
    \centering
        \includegraphics[width=12cm]{assets/figure9}
        \caption{Test con vincolo attivo di disuguaglianza.}
\end{figure} 

\newpage
\section{Minimo in 2 dimensioni: vincolo di uguaglianza}

I seguenti test sono stati effettuati applicando il vincolo di uguaglianza $R =
2Z$. Da considerare inoltre che verranno tenuti in considerazione i vincoli di
progetto imposti a priori, ovvero $R \le 1m$ e $Z \le 1m$. Siccome vale il
vincolo descritto precedentemente, il valore di $Z$ non potrà essere superiore a
$0.5m$ altrimenti la ricerca comporterebbe risultati che non sono compatibili
con i limiti progettuali imposti. \\
Le prove effettuate con diversi parametri (ma stesso punto iniziale) portano
agli stessi risultati in termine di percentuale di errore dalla funzione
desiderata.

\begin{table}[h]
    \caption{Test 1}
    \begin{center}
    \begin{tabular}{|l|l|l|l|} 
    \hline
\multicolumn{2}{|c|}{\textbf{Parametri in ingresso}} &
\multicolumn{2}{c|}{\textbf {Valori in uscita}} \\ \hline
Gradi di libertà  & \textbf{2} &  &  \\ \hline 
Passo & \textbf{0.10000} & Campioni & \textbf{61} \\ \hline 
Punto di minimo & \textbf{{[}5.000 0.800{]}} & Punto di minimo &
\textbf{{[}4.106 0.495{]}} \\ \hline 
Punto iniziale & \textbf{{[}3.000 0.300{]}} & Fobj nel punto & \textbf{0.352} \\
\hline 
Lunghezza iniziale & \textbf{0.300} & Dimezzamenti & \textbf{16} \\ \hline 
Lunghezza minima & \textbf{0.00001} & Lunghezza finale & \textbf{0.00001} \\
\hline
Flip massimi & \textbf{1000} & Flips & \textbf{239} \\ \hline 
Tolleranza & \textbf{0.00100\%} & \textbf{Perc. di errore} & \textbf{35.203\%}
\\ \hline 
    \end{tabular} 
    \end{center}
    \end{table}

\begin{figure}[H]
    \centering
        \includegraphics[width=15cm]{assets/figure11}
        \caption{Test con vincolo di uguaglianza.}
\end{figure}

\newpage
\section*{Risultati e considerazioni}

\begin{figure}[H]
    \centering
        \includegraphics[width=16cm]{assets/figure12}
        \caption{Confronto dei risultati ottenuti.}
\end{figure}

I risultati ottenuti dai test provano una solida implementazione dell'algoritmo
del simplesso per la ricerca del minimo di una funzione di costo. \\
Per tutti i risultati commentati a seguire, si è scelto di assumere una
precisione della \emph{terza cifra decimale}. La motivazione risiede nel fatto
che, per limiti di realizzazione fisica delle spire, non riusciremo a garantire
una precisione maggiore del millimetro per il raggio della spira incognita. \\ 
Nel test preliminare con due gradi di libertà, il punto di minimo ottenuto
risulta essere \\ \\
\centerline{ $I_{2} = 5.000 \pm 0.001 A$ \\ $R_{2} = 0.800 \pm 0.001m$ \\ $Z_{2}
= 0.7m$} \\ \\
con rispetto ai parametri iniziali e le condizioni di arresto scelte.
L'algoritmo si è fermato dopo 177 ribaltamenti e 14 dimezzamenti, a causa della
percentuale di errore scesa sotto la soglia che abbiamo ritenuto sufficiente
all'avvio della computazione. Se la percentuale di errore massima fosse stata
minore, i risultati ottenuti sarebbero potuti essere ancora migliori. \\
\\
L'esperimento effettuato con 3 gradi di libertà e senza vincoli ha dato dei
risultati attesi, in quanto verifichiamo che il valore della percentuale di
errore rispetto all'ideale diminuisce quando aumentiamo la definizione della
funzione, e quindi il numero di campioni in cui si divide l'asse $z$. \\ 
È da osservare la scelta delle condizioni di arresto, in quanto una scelta meno
accurata avrebbe comportato risultati inattesi, ovvero un peggioramento della
percentuale di errore all'aumentare del numero di campioni. Questo problema si
verificava nei test con lunghezza minima in ingresso e tolleranza molto alte:
l'algoritmo non riusciva a procedere abbastanza da garantire dei risultati
soddisfacenti. Per ottenere dei valori più accurati, quindi, si è deciso di
rendere molto piccoli (di qualche ordine di grandezza) i valori delle suddette
condizioni di arresto, in modo tale che l'algoritmo potesse procedere per più
passi. Il punto di minimo ottenuto quindi nell'esperimento con 3 gradi di
libertà senza vincoli è \\ \\
\centerline{ $I_{2} = 4.999 \pm 0.001 A$ \\ $R_{2} = 0.799 \pm 0.001m$ \\ $Z_{2}
= 0.700 \pm 0.001m$} \\ \\
Nel caso della ricerca del minimo in 3D con vincolo di disuguaglianza invece,
evince chiaramente l'importanza della scelta del punto iniziale. Cambiando il
punto negli esperimenti effettuati, infatti, si possono anche avere discrepanze
di 10 punti percentuali rispetto all'errore ottenuto nelle stesse condizioni, ma
punto iniziale diverso. Il punto di minimo ottenuto in questo caso è \\ \\
\centerline{ $I_{2} = 4.0926 \pm 0.001 A$ \\ $R_{2} = 0.999 \pm 0.001m$ \\ $Z_{2}
= 0.495 \pm 0.001m$} \\ \\
Il quarto ed ultimo esperimento si è invece concentrato sulla ricerca del minimo
in presenza di un vincolo di uguaglianza, che fa ridurre la dimensione dello
spazio di ricerca da 3 a 2. Con i parametri iniziali e le condizioni di arresto
scelte, si ottiene un punto di minimo di \\ \\
\centerline{ $I_{2} = 4.106 \pm 0.001 A$ \\ $R_{2} = 0.999 \pm 0.001m$ \\ $Z_{2}
= 0.495 \pm 0.001m$} \\ \\
Di particolare interesse è stata l'applicazione dei limiti di progetto
preliminari imposti ad inizio trattazione. \\ 
Siccome nè raggio nè posizione della spira possono oltrepassare la lunghezza di
$1m$, questa condizione combinata al vincolo di uguaglianza $R = 2Z$ comporta la
scelta di valori accettabili per la $Z$ fino a $0.5m$. L'esperimento ha
confermato i risultati aspettati, in quanto il valore ideale della posizione
della spira incognita è maggiore del massimo imposto dai vincoli realizzativi. È
quindi chiaro che l'algoritmo tende a muoversi verso la direzione del minimo
ideale fino a raggiungere il valore massimo ammissibile della $Z = 0.5m$.

\end{document}
